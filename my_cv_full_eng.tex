
\documentclass[12pt,a4paper,final]{moderncv}

\moderncvstyle{classic}
\moderncvcolor{grey}

\usepackage{url}

% character encoding
\usepackage[utf8]{inputenc}

% adjust the page margins
\usepackage[scale=0.75]{geometry}
% \setlength{\hintscolumnwidth}{3cm}.

\usepackage[scale=0.75]{geometry}
\setlength{\hintscolumnwidth}{3cm}

% personal data
\name{Yulia}{Yakovleva}
\title{Software engineer}
\address{}{Moscow}{Russian Federation}
\email{robolamp@ya.ru}
\social[github]{robolamp}
\extrainfo{\url{kotobank.ch/~robolamp/}}


\begin{document}
\makecvtitle

\section{Experience}

\cventry{November~2021 -- Now}{Machine learning engineer}{Descriptor.ai}{Remote/Moscow}{}{
\begin{itemize}
    \item Sound data processing with deep learning models (Python, NumPy, TensorFlow, Keras, Sound data).
\end{itemize}}

\cventry{July~2021 -- October~2021}{Machine learning engineer}{MediaZona}{Remote/Moscow}{}{
\begin{itemize}
    \item AI Text Generation (NLP, Python, NumPy, TensorFlow, Keras, Transformers, GPT):
      I worked on conditional text generation with neural networks.
      My responsibilities included both engineering/coding and interaction with non-tech employees of MediaZona on translating their 
      non-tech requirements into "tech language", finding the data and getting a feedback on text generators' work.
\end{itemize}}


\cventry{March~2018 -- May~2021}{Software engineer}{Yandex Self-Driving Cars}{Moscow}{}{
\begin{itemize}
\item Sensor diagnostics software (ROS, C++, Python, NumPy):
  I created data quality checking software modules for cameras and LiDARs.
\item Traffic lights recognition software (ROS, C++, Python, NumPy, TensorFlow, Keras).
  \begin{itemize}
  \item I worked on improvement of traffic lights recognition and tracking pipeline,
  \item learning data mining, pre-processing and datasets preparation,
  \item created, learned and deployed multiple iterations of deep neural networks,
  which are working now on hundreds of self-driving cars made by Yandex.
  \end{itemize}
\end{itemize}}

\cventry{August~2017 -- March~2018}{Software engineer}{Unemployed/Self-employed}{Moscow}{}{}

\cventry{October~2015 -- August~2017}{Robotics researcher/developer}{Institute for Information Transmission Problems RAS (Kharkevich Institute)}{Moscow}{}{
\begin{itemize}
\item Self-driving car prototype positioning and control software.
  I created or worked on the following modules:
  \begin{itemize}
  \item System launch tool to replace ROSLaunch (Python, ROS, Paramiko);
  \item Local positioning system (C++, Eigen, Kalman filters);
  \item Trajectory control system (C++, ROS);
  \item Developers' web-interface (Python, JS (Leaflet.JS, Bootstrap),  ROS);
  \item Road markup-relied localization system (C++, ROS).
  \end{itemize}
\item Initiative works in deep learning for robotics control (just for fun).
\end{itemize}}

\cventry{June~2015 -- October~2015}{Junior web-developer}{WETA Group}{Remote}{}{
Full-stack web-development
\begin{itemize}
\item Information security system web-interface: \newline
  I developed two web-applications using Django non-rel backend and JS frontend with MongoDB database;
\end{itemize}}

\cventry{July~2013 -- June~2015}{Junior control systems developer}{Modern Signal Processing and Control Technologies R\&D Laboratory}{Chelyabinsk}{}{
\begin{itemize}
\item Turboshaft engine control system development:
  \begin{itemize}
  \item I performed Turboshaft math modelling using MATLAB/Simulink,
  \item participated in control system design, test stands assembling and commissioning;
  \end{itemize}
\item Self-driving car prototype trajectory control system:
  \begin{itemize}
  \item I proposed control algorithms and performed math modelling using MATLAB/Simulink,
  \item implemented these Algorithms (C++, control unit with STM32 and NuttX RTOS),
  \item performed HIL testing using Python and NumPy and participated in field tests.
  \end{itemize}
\item I developed UAV test stand software: Scilab, interaction with National
 Instruments data acquisition system.
\end{itemize}}

\cventry{September~2012 -- June~2015}{Laboratory assistant}{South Ural State
 University}{Chelyabinsk}{}{Control systems research.}

\section{Talks}
\cventry{}{Traffic Lights in Yandex Self-Driving Cars}{Presented at Yandex Self-Driving Meetup 2019, PyLadies Moscow and PyLadies Kazan}{}{}{
A short talk in Russian about the difficulties of traffic lights recognition and
about Yandex Self-Driving Cars traffic lights recognition pipeline. \newline}
\cventry{}{Myths about Self-Driving Cars}{Presented at WTM Moscow}{}{}{
An interactive talk in Russian about self-driving cars architecture,
sensors and testing. \newline}
\cventry{}{Data mining in Yandex Self-Driving Cars}{Presented at Pytup Moscow}{}{}{
A short talk in Russian about data processing pipeline in
Yandex Self-Driving Cars project. \newline}

\section{Volunteering}
\cventry{May~2020 -- now}{Web developer/data analyst}{OVD-Info}{Remote/Moscow}{}{
  OVD-Info if an independent human rights media project.
  I'm participating in development of information collection and analysis system for OVD-Info using SQL, Python and Django.
}

\section{Education}
\cvitem{2010--2015}{
  \textbf{National Research South Ural State University}; \newline
  Computer Technologies, Control and Radio Electronics Faculty; \newline
  Automation and Control Department; \newline
  MEng with honours.
}

\cvitem{2015--2017}{
  \textbf{Moscow Institute of Physics and Technology (State University)}; \newline
  Deparment of Innovation and High Technologies; \newline
  Cognitive technologies sub-faculty; \newline
  MSc in Computer Science.
}

\section{Languages}
\cvitemwithcomment{Russian}{Native speaker}{}
\cvitemwithcomment{English}{Intermediate}{}
\cvitemwithcomment{German}{Beginner}{}
\cvitemwithcomment{Tatar}{Beginner}{}

\section{Skills}
\subsection{Main:}
\cvitem{}{C++ (Eigen), Python (Jupyter, NumPy, Keras, Sklearn), Git, ROS, Linux, Machine Learning, Computer Vision. }
\subsection{Experience with:}
\cvitem{}{Bash, C, OpenCV, JS (some outdated frameworks), Django,  Docker, \LaTeX, Dynamic systems math
modelling, Matlab/Simulink. }

\section{Pet projects}
\cventry{}{rTerm}{\url{github.com/robolamp/rTerm}}{}{}{
Fake JS-based UNIX term for my personal page.
\newline}

\cventry{}{Random three body problem bot}{\url{github.com/robolamp/3_body_problem_bot}}{}{}{
A program which is simulating the behavior of random three body system
multiple times and publishing animation of the most interesting one every
12 hours at \href{t.me/random_three_body_problem}{Telegram channel}.
\newline}

\section{Interests}
\cvitem{}{Space, alpine skiing, cross-country skiing, bicycling.}

\end{document}
