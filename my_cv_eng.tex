
\documentclass[12pt,a4paper,final]{moderncv}

\moderncvstyle{classic}
\moderncvcolor{grey}

\usepackage{url}

% character encoding
\usepackage[utf8]{inputenc}

\usepackage[scale=0.85, margin=15mm]{geometry}
% \geometry{bottom=30mm}

\setlength{\hintscolumnwidth}{3cm}

% personal data
\name{Yulia}{Yakovleva}
\title{Machine Learning Engineer}
\address{}{Amsterdam}{Netherlands}
\email{robolamp@robolamp.me} 
\social[github]{robolamp}
\social[linkedin][https://www.linkedin.com/in/robolamp/]{Yulia Yakovleva}
\extrainfo{\url{https://www.robolamp.me}}

\begin{document}
\makecvtitle

\section{Main skills}
\cvitem{Programming}{Python (PyTorch, TensorFlow, NumPy, Scikit-learn, Keras, huggingface, transformers, diffusers), C++ (Eigen), Git, Bash}
\cvitem{AI/ML}{Model optimization, Generative AI, Diffusion Models, LLMs, Deep Learning, Computer Vision, NLP}
\cvitem{Systems}{Linux, HPC, Docker, ROS}

\section{Experience}

\cventry{November~2024 -- now}{Machine Learning Engineer}{Creative Fabrica}{Amsterdam}{}{
\begin{itemize}
    \item Accelerated text-to-font pipeline from minutes to seconds, enabling scalable user-facing applications.
    \item Developed a fast, efficient pipeline for vector graphics analysis and modification.
    \item Built a custom SOTA benchmarking pipeline for generative models to evaluate and compare image generation quality.
    \item (Python, GenAI, Image generation, huggingface, diffusers, transformers, inference optimization, vector graphics generation)
\end{itemize}}


\cventry{June~2024 -- October~2024}{Head of ML Research}{AI Neko}{Remote/Amsterdam}{}{
\begin{itemize}
    \item Led short-term research on efficient LLM inference, benchmarking open-source tools (GGML, llama.cpp, transformers, quantization) and delivering prototypes in C++/Python.
\end{itemize}}

\cventry{November~2022 -- May~2024}{Machine Learning Engineer}{Stability.ai}{Remote/Amsterdam}{}{
  I worked on the following projects:
\begin{itemize}   
    \item Stable-Finetuning (Python, PyTorch, CUDA, diffusion models, HPC):
    Focused on preprocessing and training algorithm optimization, achieving multi-fold speedups while the size of model increased.
    Accelerated SAM to process large batches of images in less than 1 second
    Built a regulatory-compliant facial fine-tuning pipeline without facial keypoints detection.
    Migrated services across multiple generations of fine-tuned models and backends.
    \item Stability models API (Python, PyTorch, CUDA, diffusion models, AWS, HPC):
    Similar effort but without direct involvement into algorithms development.
    \item LLM-related project (Python, PyTorch, CUDA, LLMs, axolotl, HPC).
\end{itemize}}

\cventry{March~2022 -- September~2022}{Machine learning engineer}{Rainbow.ai}{Warsaw}{}{
\begin{itemize}
    \item Contributed to applying deep learning models for weather forecasting (Python, PyTorch, Weather RADAR data).
\end{itemize}}

\cventry{November~2021 -- February~2022}{Machine learning engineer}{Descriptor.ai}{Remote/Moscow}{}{
\begin{itemize}
    \item Delivered sentiment analysis models for voice data with strong performance (Python, NumPy, TensorFlow, Keras, Audio data).
\end{itemize}}

\cventry{July~2021 -- October~2021}{Machine learning engineer}{MediaZona}{Remote/Moscow}{}{
\begin{itemize}
    \item AI Text Generation (NLP, Python, NumPy, TensorFlow, Keras, Transformers, GPT):
      Developed conditional text generation models ahead of mainstream adoption.
      My responsibilities included both engineering/coding and interaction with non-tech employees of MediaZona on translating their 
      non-tech requirements into "tech language", finding the data and getting a feedback on text generators' work.
\end{itemize}}

\cventry{March~2018 -- May~2021}{Software engineer}{Yandex Self-Driving Cars}{Moscow}{}{
\begin{itemize}
\item Sensor diagnostics software (ROS, C++, Python, NumPy):
  I created data quality checking software modules for cameras and LiDARs.
\item Traffic lights recognition software (ROS, C++, Python, NumPy, TensorFlow, Keras).
  \begin{itemize}
  \item I worked on improvement of traffic lights recognition and tracking pipeline,
  \item learning data mining, pre-processing and datasets preparation,
  \item created, learned and deployed multiple iterations of deep neural networks,
  which are working now on hundreds of self-driving cars made by Yandex.
  \end{itemize}
\end{itemize}}

\cventry{October~2015 -- August~2017}{Robotics researcher/developer}{Institute for Information Transmission Problems RAS (Kharkevich Institute)}{Moscow}{}{
  (C++, Python, ROS,  Eigen, Computer Vision, Kalman filters)
%  I worked on self-driving car prototype positioning and control software including: system launch tool to replace ROSLaunch, positioning and control systems (C++, Python, ROS,  Eigen, Computer Vision, Kalman filters).
% \begin{itemize}
% \item Self-driving car prototype positioning and control software.
  % I created or worked on the following modules:
  % \begin{itemize}
  % \item System launch tool to replace ROSLaunch (Python, ROS, Paramiko);
  % \item Local positioning system (C++, Eigen, Kalman filters);
  % \item Trajectory control system (C++, ROS);
  % \item Developers' web-interface (Python, JS (Leaflet.JS, Bootstrap),  ROS);
  % \item Road markup-relied localization system (C++, ROS).
  % \end{itemize}
% \item Initiative works in deep learning for robotics control (just for fun).
% \end{itemize}
}

\cventry{June~2015 -- October~2015}{Junior web-developer}{WETA Group}{Remote}{}{
Full-stack web-development
% \begin{itemize}
% \item Information security system web-interface: \newline
%   I developed two web-applications using Django non-rel backend and JS frontend with MongoDB database;
% \end{itemize}
}

\cventry{July~2013 -- June~2015}{Junior control systems developer}{Modern Signal Processing and Control Technologies R\&D Laboratory}{Chelyabinsk}{}{
% \begin{itemize}
% \item Turboshaft engine control system development:
%   \begin{itemize}
%   \item I performed Turboshaft math modelling using MATLAB/Simulink,
%   \item participated in control system design, test stands assembling and commissioning;
%   \end{itemize}
% \item Self-driving car prototype trajectory control system:
%   \begin{itemize}
%   \item I proposed control algorithms and performed math modelling using MATLAB/Simulink,
%   \item implemented these Algorithms (C++, control unit with STM32 and NuttX RTOS),
%   \item performed HIL testing using Python and NumPy and participated in field tests.
%   \end{itemize}
% \item I developed UAV test stand software: Scilab, interaction with National
%  Instruments data acquisition system.
% \end{itemize}
}

\section{Publications \& Talks}
\cventry{2019}{Myths about Self-Driving Cars}{Presented at \href{https://gdgmoscow.timepad.ru/event/1101498/}{WTM Moscow}}{}{}{}
% An interactive talk in Russian about self-driving cars architecture,
% sensors and testing. \newline
% }
\cventry{2019}{Traffic Lights in Yandex Self-Driving Cars}{Presented at \href{https://medium.com/yandex-self-driving-car/yandex-self-driving-meetup-817e905b3d4a}{Yandex Self-Driving Meetup 2019}, \href{https://twitter.com/pyladies_msk/status/1145709227513733120}{PyLadies Moscow} and \href{https://x.com/pyladieskzn/status/1161901466430377985}{PyLadies Kazan}}{}{}{}
% A short talk in Russian about the difficulties of traffic lights recognition and
% about Yandex Self-Driving Cars traffic lights recognition pipeline. \newline
% }
\cventry{2020}{Data mining in Yandex Self-Driving Cars}{Presented at \href{https://events.yandex.ru/events/pytup-26-02-2020}{Pytup Moscow}}{}{}{}
% A short talk in Russian about data processing pipeline in
% Yandex Self-Driving Cars project. \newline
% }
\cventry{2020}{Method of and system for determining traffic signal state}{}{}{}{
  Artamonov, Kalyuzhny, Yakovleva
  \begin{itemize}
    \item US Patent \href{https://patents.google.com/patent/US20210201058A1}{US20210201058A1}, application at 2020.09.28, granted.
    \item European Patent \href{https://patents.google.com/patent/EP3842996A1}{EP3842996A1}, application at 2020.10.14, pending.
  \end{itemize}
}

\cventry{2023}{How does ChatGPT work?}{Presented at \href{https://youtu.be/g2u21UsAS84}{Between Brackets podcast}}{}{}{}
\cventry{2023}{What's going on in AI world}{Presented in Warsaw \href{https://gdgmoscow.timepad.ru/event/1101498/}{} }{}{}{}
% A short talk in Russian about data processing pipeline in
% Yandex Self-Driving Cars project. \newline
% }

\section{Education}
\cvitem{2010--2015}{
  \textbf{National Research South Ural State University}; \newline
  Computer Technologies, Control and Radio Electronics Faculty; \newline
  Automation and Control Department; \newline
  MEng with honours.
}

\cvitem{2015--2017}{
  \textbf{Moscow Institute of Physics and Technology (State University)}; \newline
  Department of Innovation and High Technologies; \newline
  Cognitive technologies sub-faculty; \newline
  MSc in Computer Science.
}

\section{Volunteering}
\cventry{May~2020 -- now}{Technical volunteer (Backend \& Machine Learning)}{OVD-Info}{Remote}{}{
  OVD-Info is an independent human rights media project.
  \begin{itemize}
    \item Developed and maintained backend systems for information collection and analysis (SQL, Python and Django).
    \item Delivered ad-hoc data analysis and built pipelines for online text data processing and monitoring (Python, NLP, LLMs, PyTorch). 
  \end{itemize}
}
\cventry{Jan~2021 -- now}{Technical volunteer}{Memorial}{Remote}{}{
  Memorial is one of the oldest Russian human rights NGOs.
  \begin{itemize}
    \item Provided technical and administrative support for internal systems and documentation.
  \end{itemize}
}

% \section{Languages}
% \cvitemwithcomment{Russian}{Native speaker}{}
% \cvitemwithcomment{Tatar}{Native speaker}{}
% \cvitemwithcomment{English}{Fluent}{}
% \cvitemwithcomment{Polish}{Beginner}{}
% \cvitemwithcomment{German}{Beginner}{}
% \cvitemwithcomment{Dutch}{Beginner}{}

\section{Pet projects}
\cventry{}{rTerm}{\url{github.com/robolamp/rTerm}}{}{}{
Fake JS-based UNIX term for my personal page.
\newline}

\cventry{}{Random three body problem bot}{\url{github.com/robolamp/3_body_problem_bot}}{}{}{
A program which is simulating the behavior of random three body system
multiple times and publishing animation of the most interesting one every
12 hours at \href{t.me/random_three_body_problem}{Telegram channel}.
\newline}

% \section{Interests/hobbies}
% \cvitem{}{wildife photography, alpine skiing, cross-country skiing, books, jogging}


\end{document}
